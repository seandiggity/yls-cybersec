\documentclass[11pt, a4paper]{article}
%\usepackage{geometry}
\usepackage[inner=1.5cm,outer=1.5cm,top=2.5cm,bottom=2.5cm]{geometry}
\pagestyle{empty}
\usepackage{graphicx}
\usepackage{fancyhdr, lastpage, bbding, pmboxdraw}
\usepackage[usenames,dvipsnames]{color}
\definecolor{darkblue}{rgb}{0,0,.6}
\definecolor{darkred}{rgb}{.7,0,0}
\definecolor{darkgreen}{rgb}{0,.6,0}
\definecolor{red}{rgb}{.98,0,0}
\usepackage[colorlinks,pagebackref,pdfusetitle,urlcolor=darkblue,citecolor=darkblue,linkcolor=darkred,bookmarksnumbered,plainpages=false]{hyperref}
\renewcommand{\thefootnote}{\fnsymbol{footnote}}

\pagestyle{fancyplain}
\fancyhf{}
\rhead{ \fancyplain{}{Cybersecurity -- LAW 20310, Fall 2018} }
%\chead{ \fancyplain{}{} }
%\rhead{ \fancyplain{}{\today} }
%\rfoot{\fancyplain{}{page \thepage\ of \pageref{LastPage}}}
\fancyfoot[RO, LE] {page \thepage\ of \pageref{LastPage} }
\thispagestyle{plain}

%%%%%%%%%%%% LISTING %%%
\usepackage{listings}
\usepackage{caption}
\DeclareCaptionFont{white}{\color{white}}
\DeclareCaptionFormat{listing}{\colorbox{gray}{\parbox{\textwidth}{#1#2#3}}}
\captionsetup[lstlisting]{format=listing,labelfont=white,textfont=white}
\usepackage{verbatim} % used to display code
\usepackage{fancyvrb}
\usepackage{acronym}
\usepackage{amsthm}
\VerbatimFootnotes % Required, otherwise verbatim does not work in footnotes!

%% symbols
\usepackage{bbding}
\usepackage{pifont}
\usepackage{wasysym}
\usepackage{amssymb}

\definecolor{OliveGreen}{cmyk}{0.64,0,0.95,0.40}
\definecolor{CadetBlue}{cmyk}{0.62,0.57,0.23,0}
\definecolor{lightlightgray}{gray}{0.93}

\lstset{
%language=bash,                          % Code langugage
basicstyle=\ttfamily,                   % Code font, Examples: \footnotesize, \ttfamily
keywordstyle=\color{OliveGreen},        % Keywords font ('*' = uppercase)
commentstyle=\color{gray},              % Comments font
numbers=left,                           % Line nums position
numberstyle=\tiny,                      % Line-numbers fonts
stepnumber=1,                           % Step between two line-numbers
numbersep=5pt,                          % How far are line-numbers from code
backgroundcolor=\color{lightlightgray}, % Choose background color
frame=none,                             % A frame around the code
tabsize=2,                              % Default tab size
captionpos=t,                           % Caption-position = bottom
breaklines=true,                        % Automatic line breaking?
breakatwhitespace=false,                % Automatic breaks only at whitespace?
showspaces=false,                       % Dont make spaces visible
showtabs=false,                         % Dont make tabls visible
columns=flexible,                       % Column format
morekeywords={__global__, __device__},  % CUDA specific keywords
}

%%%%%%%%%%%%%%%%%%%%%%%%%%%%%%%%%%%%
\begin{document}
\begin{center}
{\LARGE \textsc{Cybersecurity}} \\
\vspace{.2cm}
LAW 20310, Fall 2018
\begin{center}
\rule{6.5in}{0.4pt}
\begin{minipage}[t]{.75\textwidth}
\begin{tabular}{llcccll}
\textbf{Time:} Tuesday 10:00am--12:00 noon & \textbf{Place:} 40 Ashmun (Baker Hall), Rm 120 \\
\end{tabular}
\end{minipage}
\rule{6.5in}{0.4pt}
\end{center}
\setlength{\unitlength}{1in}
\renewcommand{\arraystretch}{2}
\end{center}

\noindent\textbf{Instructors:}
\begin{itemize}
\item[] Scott Shapiro, scott.shapiro@yale.edu \& Sean O'Brien, sean.obrien@yale.edu
\item[] Assisted By -- Laurin Weissinger, Cybersecurity Fellow, laurin.weissinger@yale.edu
\end{itemize}

\noindent\textbf{Office Hours:}
\begin{itemize}
\item[] Sean O’Brien -- Thursday, 4:30pm--5:30pm, Baker Hall 438
\item[] Scott Shapiro -- Monday, 4:30pm--5:30pm, SLB 325
\item[] Laurin Weissinger -- Wednesday, 11:00am--12:00 noon, Baker Hall 438
\end{itemize}

\noindent\textbf{Course Websites:} \begin{enumerate}
\item Yale Canvas -- \url{https://yale.instructure.com/courses/38230}
\item More Resources -- \url{https://github.com/seandiggity/yls-cybersec}
\end{enumerate}

\vskip.1in
\noindent\textbf{Description and Objectives:}  This course is an introduction to cybersecurity, privacy, anonymity, and cryptography via hands-on activities. Students will learn cybersecurity and networking concepts so that they may better engage issues at the policy and regulatory level.

\vskip.2in
\noindent\textbf{Technical Requirements:}
A laptop computer is required for each class.  We will be utilizing a Command Line Interface (CLI) on each laptop.  Students will communicate and control Raspberry Pi mini-computers via the Secure Shell (SSH) protocol.  Please install the software below on the laptop you will use in class. 
\begin{itemize}
\item Hyper -- \url{https://hyper.is} (Command Line Interface / Terminal Emulator)
\item Filezilla Client -- \url{https://filezilla-project.org} (SSH / SFTP Client)
\item Atom -- \url{https://atom.io} (Text Editor)
\item Git for Windows -- \url{https://gitforwindows.org} (Windows users only, may be required for SSH)
\end{itemize}

\vskip.1in
\noindent\textbf{Course Requirements:}
\begin{itemize}
\item \textbf{Attendance} -- It is very important to attend each class. Attendance is mandatory.
\item \textbf{Homework} -- Most classes conclude with a take-home assignment. It will be graded as \ding{51}+, \ding{51}, or \ding{51}--
\item \textbf{Final Project} -- Video demonstration of three attacks/hacks with accompanying written description. Due by the last day of class.
\item \textbf{Final Exam} -- Take home exam consisting of two questions, open-book, 24 hours to complete.
\item \textbf{Grading} -- Homework (33\%); Final Project (33\%); Final Exam (33\%).
\end{itemize}

\pagebreak
\noindent \textbf{\Large{Course Outline:}}

\vspace*{.2in}
\noindent\textbf{Week 1 -- Practical Cybersecurity}
\begin{enumerate}
\item Our Approach
\item Digital Self-Defense
\item Classroom Network Diagram
\item Command Line Interface (CLI)
\item Raspberry Pi Assembly
\end{enumerate}

\vspace*{.1in}
\noindent\textbf{Week 2 -- Get to Know Your Mini-Computer}
\begin{enumerate}
\item Command Line Basics
\item Controlling Your Raspberry Pi via SSH
\item Client/Server Model
\item The Filesystem Tree
\item Edit a File
\end{enumerate}

\vspace*{.1in}
\noindent\textbf{Week 3 -- Operating Systems}
\begin{enumerate}
\item Admin / Root Access
\item The Kernel
\item Userspace
\item Processes
\item Rootkits
\end{enumerate}

\vspace*{.1in}
\noindent\textbf{Week 4 -- Ownership \& Permissions}
\begin{enumerate}
\item Permissions as a Structural Design for Security
\item Creating Users and Groups
\item Principle of Least Privilege
\item Sandboxing \& Isolation
\item Privilege Escalation Attacks
\end{enumerate}

\vspace*{.1in}
\noindent\textbf{Week 5 -- Normative Structure of a Network}
\begin{enumerate}
\item IP Address, Physical Address
\item Networking Models \& Protocols (OSI Model)
\item Internet Infrastructure
\item Request/Response via the Web
\item Distributed Denial-of-Service (DDoS)
\end{enumerate}

\vspace*{.1in}
\noindent\textbf{Week 6 -- Network Attacks}
\begin{enumerate}
\item Domain Names
\item DNS Poisoning
\item Changing Your Pi's Network Identification
\item Ports \& Firewalls
\item Man-in-the-Middle Attacks (MITM)
\end{enumerate}

\vspace*{.1in}
\noindent\textbf{Week 7 -- Secrecy \& Encryption}
\begin{enumerate}
\item Obfuscation \& Hashes
\item Public/Private Keys
\item HTTP Encryption (SSL/TLS)
\item E-mail Encryption (PGP/GPG)
\item Weaknesses
\end{enumerate}

\vspace*{.1in}
\noindent\textbf{Week 8 -- Information Security}
\begin{enumerate}
\item Data as a Toxic Asset %This needs explaining  %% SO - I'm using Bruce Schneier's definition here - https://www.schneier.com/blog/archives/2016/03/data_is_a_toxic.html  There's also a Berkman paper with Schneier and Zittrain which is a good "more reading" for the class.  I can be quick on this and/or move it to another section, but I think it's a good analogy for students.
\item What is InfoSec?
\item Confidentiality
\item Integrity
\item Availability
\end{enumerate}

\vspace*{.1in}
\noindent\textbf{Week 9 -- Anonymity \& The Dark Web}
\begin{enumerate}
\item Onion Routing (Tor)
\item Censorship Circumvention
\item Tor Config on FreedomBox
\item Sharing Files Anonymously
\item Cryptomarkets
\end{enumerate}

\vspace*{.1in}
\noindent\textbf{Week 10 -- Cybercrime}
\begin{enumerate}
\item Cryptocurrency \& Transactions %% SO - just a note, I'm just going to call Bitcoin a ledger, Ethereum a ledger with "smart contracts" built in, and talk about how it reduces intermediaries in money transfers.  Nothing about the actual tech in detail.
\item Ransomware
\item Fraud \& Phishing
\item Data Breaches
\item Challenges for Attack Attribution
\end{enumerate}

\vspace*{.1in}
\noindent\textbf{Week 11 -- Chains of Trust}
\begin{enumerate}
\item Trusted Software Distribution
\item Software Verification
\item Hardware Assurance
\item Free \& Open-Source Software
\item Static Analysis
\end{enumerate}

\vspace*{.1in}
\noindent\textbf{Week 12 -- Penetration Testing}
\begin{enumerate}
\item Cross-Site Scripting (XSS)
\item SQL Injection Attacks
\item Delivering Payloads
\item Metasploit Framework
\item Using Metasploit %% SO - We'll have to choose the best + most feasible examples here.
\end{enumerate}

\vspace*{.1in}
\noindent\textbf{Week 13 -- Threat Modeling}
\begin{enumerate}
\item Risks and Vulnerabilities
\item Zero Day Attacks
\item Attack Scenarios
\item Mitigation
\item Operational Security (OPSEC)
\end{enumerate}

%%%%%% THE END 
\end{document} 